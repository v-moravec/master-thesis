%! Author = Vojta
%! Date = 21.1.2024

\chapter{Úvod}
V moderním světě vývoje webových aplikací se neustále zvyšuje důraz na efektivitu, opakovanou použitelnost a rychlost implementace. Tyto vlastnosti jsou důležité zejména v kontextu stále rostoucích požadavků na kvalitu uživatelských rozhraní (UI) a uživatelského zážitku (UX). Vývojáři se proto stále častěji obracejí k využití UI knihoven, které jim poskytují předdefinované komponenty a nástroje pro rychlé a efektivní vytváření komplexních a esteticky příjemných rozhraní. Tyto knihovny jsou důležitou součástí vývoje aplikací, protože usnadňují implementaci složitých UI prvků.

Nicméně jeden z problémů dnešních knihoven je ten, že vývojáři nemohou jednoduše modifikovat nebo rozšiřovat existující komponenty. To může vést k problémům při vytváření unikátních a přizpůsobených uživatelských rozhraní.

Tato práce se zabývá analýzou různých aspektů UI knihoven, včetně jejich designu, funkcí. Cílem je poskytnout ucelený pohled na UI knihovny, jejich výhody, výzvy a \emph{best practices} pro jejich využití ve webovém vývoji.


\chapter{Cíl}
Hlavním cílem této práce je vytvořit sbírku znovupoužitelných a snadno modifikovatelných komponent pro webové aplikace postavené na \emph{frameworku} Nuxt. Tato sbírka komponent má usnadnit počáteční fázi vývoje, nabídnout možnosti rozšíření a přizpůsobení podle individuálních potřeb. Práce se nezaměřuje pouze na vývoj samotných komponent, ale zahrnuje také jejich podrobnou dokumentaci a nástroje, které podporují všechny fáze tvoření aplikace – od návrhu až po nasazení.

V teoretické části práce je popsán výběr zvolených technologií a analýza existujících řešení, která mohou sloužit jako inspirace pro tvorbu komponent. Tato analýza zahrnuje detailní přehled různých přístupů k vývoji UI knihoven, jejich výhod a nevýhod, a posouzení, jak tyto přístupy ovlivňují efektivitu a flexibilitu vývoje. Důkladně jsou prozkoumány existující knihovny komponent, jako jsou například Nuxt UI a Radix UI, a jejich přínosy a omezení v kontextu konkrétních projektů. Rovněž jsou porovnány dva hlavní přístupy k integraci komponent do projektů – distribuce pomocí balíčků versus vlastnění kódu.

Praktická část se věnuje návrhu, vývoji a testování sbírky, dokumentace a možnostem budoucího rozšíření. Proces návrhu zahrnuje tvorbu návrhů jednotlivých komponent. Vývoj komponent je prováděn s důrazem na dodržování \emph{best practices} a použití moderních nástrojů a technologií, jako jsou Vue, Nuxt, TypeScript a Tailwind CSS. V této části jsou rovněž popsány konkrétní techniky používané při implementaci. Testování zahrnuje uživatelské testování, aby bylo zajištěno, že komponenty splňují stanovené požadavky a jsou snadno použitelné.
