%! Author = Vojta
%! Date = 21.1.2024

\chapter{Úvod}
UI knihovny jsou nezbytné nástroje ve světě webového vývoje. Poskytují sadu předdefinovaných komponent, které vývojářům umožňují rychle
a efektivně vytvářet atraktivní a funkční uživatelská rozhraní. Tyto knihovny jsou důležitou součástí vývoje aplikací, protože
usnadňují implementaci složitých UI prvků.

Tato práce se zabývá analýzou různých aspektů UI knihoven, včetně jejich designu, funkcí, a toho, jak tyto knihovny podporují různé části
vývoje aplikace. Cílem je poskytnout ucelený pohled na UI knihovny, jejich výhody, výzvy a \emph{best practices} pro jejich využití ve webovém vývoji.


\chapter{Cíl}
Cílem této práce je vytvořit sadu znovupoužitelných a snadno modifikovatelných komponent pro webové aplikace postavené na frameworku Nuxt.
Tato kolekce komponent má usnadnit počáteční fázi vývoje, nabízet možnosti rozšíření a přizpůsobení podle individuálních potřeb.
Práce se neomezuje pouze na vývoj samotných komponent, ale zahrnuje také jejich podrobnou dokumentaci a nástroje, které podporují všechny fáze
tvoření aplikace.

Nejdříve je v práci připravena teoretická část, která se zaměřuje na popis zvolených technologií a dalších řešení. Dále se práce zabývá návrhem 
a vývojem komponent a jejich dokumentace. Nakonec je nastíněn možný budoucí vývoj a rozšíření.
