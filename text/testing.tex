%! Author = Vojta
%! Date = 21.1.2024

\chapter{Testování}

\section{Úvod}
Testování je zásadní součástí vývoje jakéhokoliv softwarového produktu. Zajišťuje nejen funkčnost a bezpečnost aplikací, ale také zlepšuje uživatelskou zkušenost a usnadňuje adaptaci produktu koncovými uživateli. V rámci této diplomové práce se zaměřujeme na testování dokumentace, procesů instalace a praktického nasazení kolekce komponent ve vývojářských projektech. Cílem je ověřit, jak intuitivní a přístupné jsou tyto aspekty pro vývojáře různých úrovní zkušeností.

Specifickým cílem testování v této práci je posoudit, zda dokumentace poskytuje jasné a snadno dostupné informace potřebné pro efektivní využití kolekce, zda jsou instrukce k instalaci dostatečně srozumitelné a zda je implementace komponent do různých typů projektů co možná nejsnadnější. Tato kapitola popisuje metodiku testování, vybrané testovací scénáře, a představuje výsledky získané z provedených testů.

Testování bylo zaměřeno na praktické aplikace v reálných vývojářských prostředích, což umožnilo shromáždit autentické zpětné vazby a poznatky, jež reflektují skutečné použití kolekce ve vývojovém procesu. Výsledky těchto testů poskytují cenné informace pro další vývoj a zlepšení dokumentace a uživatelské přívětivosti kolekce. Tato kapitola klade důraz na důkladné vyhodnocení shromážděných dat a jejich interpretaci.

\section{Metodika testování}
Metodika byla zvolena, aby zajistila komplexní a objektivní zhodnocení testované kolekce z různých perspektiv a umožnila tak formulovat podrobné a praktické doporučení pro další vývoj a zlepšování dokumentace a uživatelské přívětivosti kolekce.

\subsection{Výběr respondentů}
Pro účely tohoto testování byli vybráni tři vývojáři, každý s jinou úrovní zkušeností – od juniora přes mid-level až po senior vývojáře. Tento rozmanitý výběr zajišťuje, že zpětná vazba a zjištěné výsledky budou reflektovat široké spektrum potenciálních uživatelů kolekce. Každý respondent byl vybrán na základě jeho předchozího zapojení do podobných projektů a zkušeností s používáním \emph{frameworku} Nuxt/Vue.

\subsection{Prostředí testování}
Testování bylo provedeno ve dvou režimech - online a osobně. Každý respondent pracoval na vlastním počítači, což umožnilo simulaci skutečných pracovních podmínek. Tento přístup pomohl zajistit, že všechny zjištěné výsledky jsou přímo spojeny s testovanými aspekty kolekce, bez rušivých vlivů z externího prostředí.

\subsection{Proces testování}
Každý respondent dostal sérii úkolů odpovídajících jednotlivým testovacím scénářům. respondentům byly poskytnuty instrukce, které zahrnovaly specifické kroky a očekávané výstupy pro každý úkol. Účastníci byli požádáni, aby nahlas dokumentovali svůj postup a jakékoliv problémy nebo otázky, které během testování vznikly. Záznam byl prováděn pomocí nahrávání obrazovky a zapisování poznámek.

\subsection{Zpětná vazba}
Po dokončení testovacích úkolů byla od respondentů vyžádána zpětná vazba prostřednictvím rozhovorů a diskuzí nad problémy. Cílem bylo, aby bylo možné získat hlubší vhledy do uživatelské zkušenosti a identifikovaly klíčové oblasti pro zlepšení. Otázky zahrnovaly hodnocení obtížnosti úkolů, kvality informací v dokumentaci a celkového vnímání kolekce. Díky zpětné vazbě bylo možné iterovat a upravovat dokumentaci a procesy instalace tak, aby lépe vyhovovaly potřebám uživatelů.

\section{Testovací scénáře}
Každý ze scénářů byl navržen tak, aby testoval specifické aspekty kolekce a zajišťoval přesné a relevantní zpětné vazby. Výsledky těchto testů poskytují důležité informace pro další vylepšení uživatelské přívětivosti a efektivity dokumentace a instalace.

\begin{enumerate}
  \item Scénář: Proces instalace
  \begin{itemize}
    \item Cíl: Ověřit, zda jsou instrukce k instalaci jasné a snadno pochopitelné.
    \item Kroky úkolu
    \begin{enumerate}
      \item Otevřete si dokumentaci a krátce se seznamte s hlavními principy kolekce.
      \item Inicializujte nový Nuxt projekt pomocí příkazu nuxi init my-project v příkazové řádce.
      \item Inicializujte kolekci ve vytvořeném projektu.
    \end{enumerate}
    \item Očekávaný výsledek: Instrukce k instalaci by měly být formulovány tak, aby vývojáři mohli kolekci nainstalovat bez potřeby vyhledávání dodatečných informací nebo žádání o vnější pomoc. Úspěšná instalace bez komplikací a dotazů ukazuje, že dokumentace je dobře strukturovaná a efektivní.
  \end{itemize}
  \item Scénář: Přehlednost dokumentace
  \begin{itemize}
    \item Cíl: Ověřit, jak snadno mohou vývojáři najít informace, které potřebují.
    \item Kroky úkolu
    \begin{enumerate}
      \item Najděte v dokumentaci instrukce pro použití komponenty tlačítko a použijte ji.
      \item Najděte v dokumentaci instrukce pro použití komponenty tabulka a použijte ji.
      \item Najděte v dokumentaci instrukce pro použití komponenty sidebar a použijte ji.
    \end{enumerate}
    \item Očekávaný výsledek: Vývojáři by měli být schopni rychle a intuitivně najít potřebné informace, aniž by museli procházet nepřehlednou nebo zmatenou dokumentaci. Rychlost a snadnost nalezení informací jsou klíčové metriky pro úspěch v tomto scénáři.
  \end{itemize}
  \item Scénář: Použití sbírky ve vlastních projektech
  \begin{itemize}
    \item Cíl: Zjistit, jak snadné je používat komponenty kolekce v různých projektech.
    \item Kroky úkolu
    \begin{enumerate}
      \item Vytvořte základní strukturu \emph{landing page} pro imaginárního klienta s následujícími sekcemi.
      \item \emph{Hero sekce}: Obsahuje slogan, tlačítka s odkazy a \emph{carousel} s obrázky.
      \item Reference: Sekce pod \emph{hero sekcí}, která ukazuje reference na práci klienta.
      \item Doplňující sekce: Vytvořte sekci dle vlastní představy, která by mohla být vhodným doplněním mezi referencemi a kontaktem.
      \item Kontaktní formulář: Na konci stránky.
    \end{enumerate}
    \item Očekávaný výsledek: Vývojáři by měli být schopni bez problémů použít vybrané komponenty a bloky a integrovat je do svých projektů. Snadnost integrace a flexibilita komponent při použití v různých kontextech a projektech jsou klíčové ukazatele úspěšnosti.
  \end{itemize}
\end{enumerate}

\section{Výsledky testování}

\subsection{Respondent 1}

\paragraph{Nalezené problémy}
\begin{enumerate}
  \item Použitelnost sidebaru: Respondent nevěděl, jak do sidebaru přidat svůj obsah.
  \item Typy dat z komponent: Bylo zjištěno, že export typů dat z komponent je důležitý pro jejich snadné použití, avšak u některých komponent chyběl.
  \item \emph{Container} komponenta: Absence komponenty pro kontejner ztížila respondentovi správu layoutu aplikací.
  \item Typescript závislost: Objevil se \emph{error}, kdy v konzoli bylo zobrazeno, že chybí Typescript závislosti. Respondent byl nucen ručně nainstalovat chybějící balíčky, ale vzhledem k jasné zprávě v konzoli to nebyl velký problém. Prozatím není nutné přidávat závislost automaticky, pokud by se problém opakoval, mohlo by to být užitečné.
  \item Carousel a \emph{hot reload}: Upozornění na \emph{hot reload} pro \emph{carousel} komponentu bylo považováno za užitečné, protože pomáhá v případě, že dojde k dynamickým změnám obsahu.
  \item Odstranění odkazů: Testování odhalilo, že přítomnost odkazů v ukázkách (např. blocích) může vést k záměně s dokumentací.
  \item Rozdíl mezi ukázkou a implementací komponenty: Respondent byl zmaten při rozhodování, kdy použít ukázkový kód a kdy se jedná o kód komponenty.
  \item Rozlišení mezi znovupoužitelnými a specifickými bloky: Respondent měl problémy rozlišovat mezi bloky, které jsou znovupoužitelné a těmi, které vyžadují úpravy.
\end{enumerate}

\paragraph{Pozitivní zjištění}
\begin{enumerate}
  \item Automatická instalace závislostí: Automatizace instalace závislostí byla hodnocena velmi kladně, neboť respondentovi usnadnila začlenění komponent do jeho projektu.
  \item Efektivita: Dokázal si představit, že základní UI u nových projektů dokáže vytvořit za krátkou dobu, což naznačuje, že kolekce je efektivní a snadno použitelná.
  \item Přirovnání k jednoduchosti použití Wordpress šablon, s rozdílem v moderním stacku.
\end{enumerate}

\paragraph{Provedené změny}
\begin{enumerate}
  \item Zvýraznění bloků v dokumentaci - představení bloků, jak se mají používat a jaké jsou jejich dva typy.
  \item Úprava stránky bloků - přidání štítků, které označují, zda se jedná o znovupoužitelný blok nebo blok, který vyžaduje úpravy.
  \item Přidání ukázky, jak přidat obsah do sidebaru.
  \item Přidání upozornění na restart development serveru. \footnote{Potřeba restartovat development server po instalaci je někdy potřeba, jindy zase ne. Kód se může rozbít kvůli přidání nových závislostí napozadí nebo kvůli jiným změnám. Upozornění je tedy užitečné ve chvíli, kdy uživatel narazí na neočekávané chování, které může být pomocí tohoto kroku vyřešeno.}
\end{enumerate}


\subsection{Respondent 2}

\paragraph{Nalezené problémy}
\begin{enumerate}
  \item Nebylo jasné odkud volat příkaz pro přidání komponent.
  \item Nebylo jasné, že se přidává do projektu zdrojový kód komponent - respondent si myslel, že se komponenty přidávají do složky node\_modules.
  \item Nedostatečné informace v rámci CLI. Např. chybějící informace o tom, že možná bude potřeba restartovat development server po instalaci, aby vše fungovalo správně.
  \item Respondentovi dělalo problém najít bloky i přes přidanou informaci v dokumentaci.
  \item V dokumentaci chybí informace o typography komponentách.
  \item Nejasné jak se mají používat bloky - respondent blok přidal pomocí CLI, ale místo jeho použití ještě manuálně vykopíroval kód do projektu.
\end{enumerate}

\paragraph{Pozitivní zjištění}
\begin{enumerate}
  \item Dokumentace byla hodnocena jako přehledná a snadno čitelná.
  \item Respondent ocenil myšlenku přístupu ke zdrojovému kódu.
\end{enumerate}

\paragraph{Provedené změny}
\begin{enumerate}
  \item Do návodu na instalaci přidána informace o volání CLI z kořenového adresáře projektu.
  \item Zvýraznění bloků v dokumentaci - přidány odkazy přímo na stránky konkrétních bloků.
  \item Přidány informace k zdrojovým kódům v dokumentaci - k čemu slouží a jak je využívat.
  \item Úprava textových informací v rámci CLI - přidání upozornění na restart development serveru a hlášky o přidání zdrojového kódu v rámci složek projektu.
  \item Přidání typography komponent do dokumentace.
\end{enumerate}


\subsection{Respondent 3}

\paragraph{Nalezené problémy}
\begin{enumerate}
  \item Hlášky o použití u zdrojových kódu někdy nejsou vidět, protože jsou pod dlouhým kódem.
  \item Závislost na Typescriptu by měla být automaticky přidána, protože jinak byl respondent zmaten a myslel si, že chyba je v kódu komponenty.
  \item Je nezvyklé vzít si celou sekci hotovou - je pro něj automatické si vytvořit vlastní komponentu.
  \item Respondent nenavšívil stránku s bloky, protože se proklikával přímo na konkrétní stránky z dokumentace.
\end{enumerate}

\paragraph{Pozitivní zjištění}
\begin{enumerate}
  \item Respondent ocenil jednoduchost a snadnost použití kolekce.
  \item Základní struktura kolekce byla hodnocena jako přehledná a snadno použitelná.
\end{enumerate}

\paragraph{Provedené změny}
\begin{enumerate}
  \item Přesunutí informací nad zdrojový kód v dokumentaci.
  \item Přidání poznámky o nutnosti instalace Typescript závislostí u komponenty Carousel.
  \item Přidání informací o blocích na \emph{landing page}.
\end{enumerate}

\section{Závěr}
Testování sbírky znovupoužitelných Nuxt komponent přineslo cenné poznatky o jejích silných stránkách i oblastech pro zlepšení. Celkově byla dokumentace hodnocena pozitivně, zejména co se týče její přehlednosti a srozumitelnosti, avšak určité části mohou být dále vylepšeny, zejména v oblasti ukázek možností komponent.

Výsledky testování ukázaly, že jednoduchá rozšiřitelnost a flexibilita komponent jsou důležitými prvky kolekce, které vývojáři ocenili. Umožňují rychlou integraci a usnadňují vytvoření webových stránek s použitím komponent a bloků.

Na základě získané zpětné vazby byly provedeny některé úpravy, jako například přidání informací o volání CLI z kořenového adresáře projektu, zvýraznění bloků v dokumentaci, a přidání poznámky o nutnosti instalace Typescript závislostí u některých komponent.

Tato testovací fáze poskytla informace pro další vývoj a zlepšování dokumentace a uživatelské přívětivosti kolekce. Bude důležité pokračovat v implementaci doporučených změn a sbírat další zpětnou vazbu, aby se zajistilo, že sbírka komponent zůstane užitečným a efektivním nástrojem.
