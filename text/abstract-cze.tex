%! Author = Vojta
%! Date = 21.1.2024

Tato práce se zaměřuje na návrh a vývoj sbírky znovupoužitelných komponent pro webové aplikace využívající framework Nuxt.
Cílem je vytvořit kolekci komponent, které zjednoduší začátek vývoje a zároveň budou jednoduše rozšiřitelné a upravitelné.
Práce se nezaměřuje pouze na samotné komponenty, ale také na jejich dokumentaci a další nástroje podporující celý životní
cyklus aplikace od návrhu až po nasazení.

K dispozici jsou i rozsáhlejší sekce, skládající se z těchto komponent. Vývojáři tak mohou využít již předpřipravená řešení,
která si mohou dle potřeby upravit, což značně urychluje proces vývoje.

Pro designéry je připravena knihovna ve Figmě, umožňující jim poskytovat vývojářům návrhy, které lze replikovat v přesné
shodě s originálem. Vývojáři mohou využít několik způsobů instalace komponent a to sice pomocí kopírování kódu nebo pomocí CLI.

% Customizace designu
