%! Author = Vojta
%! Date = 21.1.2024

\chapter{Implementace}

Začal jsem vytvořením projektu pomocí \mintinline{console}|npx nuxi@latest init content-app -t content|. Tento příkaz vytvoří
projekt s připravený Nuxt Content modulem. Dále jsem přidal potřebné moduly pro vývoj, jako je například Tailwind, Nuxt Icon,
Nuxt Color Mode a v neposlední řadě Nuxt SEO. Pro větší konzistenci kódu jsem přidal i ESLint a Prettier.

\clearpage

\section{Nastavení Tailwindu}

Nejdříve jsem se potýkal s chybějícími styly. Chtěl jsem to řešit pomocí \emph{safelistu}, ale to mi nepřišlo jako ideální řešení.
Po chvíli jsem zjistil, že lze specifikovat, které složky má při buildu Tailwind procházet a hledat tak potřebné styly. Tailwind totiž
využívá tree shakingu, takže výsledný balíček obsahuje pouze styly, které jsou použité. Toto nastavení jsem přidal do \mintinline{js}|tailwind.config.js|.
Konkrétně se jednalo o řádek 7 v ukázce \ref{lst:tailwind-config}.

\begin{listing}[h]
    \caption{Konfigurační soubor pro Tailwind}
    \label{lst:tailwind-config}
    \begin{minted}{js}
import type { Config } from 'tailwindcss'
import typography from '@tailwindcss/typography'
import containerQueries from '@tailwindcss/container-queries'

export default <Partial<Config>>{
    darkMode: 'class',
    content: ['components/**/*.{vue,ts}', 'layouts/**/*.vue', 'pages/**/*.vue'],
    theme: {
        extend: {
            colors: {
                background: 'hsl(var(--background))',
                primary: {
                    DEFAULT: 'hsl(var(--primary))',
                    contrast: 'hsl(var(--primary-contrast))',
                },
                secondary: {
                    DEFAULT: 'hsl(var(--secondary))',
                    contrast: 'hsl(var(--secondary-contrast))',
                },
                disabled: {
                    DEFAULT: 'hsl(var(--disabled))',
                    contrast: 'hsl(var(--disabled-contrast))',
                },
                border: 'hsl(var(--border))',
            },
            borderRadius: {
                sm: 'calc(var(--radius) - 4px)',
                md: 'calc(var(--radius) - 2px)',
                lg: 'var(--radius)',
                xl: 'calc(var(--radius) + 4px)',
                '2xl': 'calc(var(--radius) + 8px)',
                '3xl': 'calc(var(--radius) + 16px)',
            },
        },
    },
    plugins: [typography, containerQueries],
}
    \end{minted}
\end{listing}

\section{Zobrazení kódu}

Nuxt Content má zabudovanou možnost zobrazení kódu. \cite{NuxtContentCodeHighlighting} Potřeboval jsem vymyslet, jak zobrazit kód bez jeho duplikace.
V tom mi pomohl zdrojový kód knihovny NuxtUI. Tato knihovna využívá principu vygenerování endpointů pro zobrazení zdrojového kódu
přímo ze souborů s komponentami, popřípadě ze souborů s ukázkami použití. \cite{NuxtUISourceCodeModule}

\section{Deploy}
Dokumentace se automaticky nasadí pomocí Github Actions na Cloudflare Pages. Nastavení je velmi jednoduché, stačí mít účet na Githubu a Cloudflare,
po vytvoření repozitáře jej spojit s projektem na Clouflare Pages a při každém pushi do větve \mintinline{console}|main| se dokumentace automaticky nasadí.



% \subsection{Body pro implementaci}
% \begin{itemize}
%     \item Dokumentace
%     \item Komponenty
% \end{itemize}

