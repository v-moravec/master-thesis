%! Author = Vojta
%! Date = 21.1.2024

\chapter{Možnosti rozšíření}

V této kapitole se zaměříme na možnosti rozšíření sbírky znovupoužitelných Nuxt komponent. Po úspěšné implementaci základní sady komponent je důležité zvážit, jakým způsobem lze tuto sbírku dále rozšiřovat a přizpůsobovat novým požadavkům.

Budou zde prozkoumány možnosti přidávání nových komponent, bloků a funkcionalit, které mohou zvýšit hodnotu a využitelnost celé sbírky. Dále se bude diskutovat o integraci dalších nástrojů a technologií, které mohou usnadnit vývoj a zlepšit uživatelskou zkušenost.

Cílem této kapitoly je poskytnout ucelený pohled na možnosti rozšíření a ukázat, jak lze existující sbírku komponent nadále zdokonalovat a přizpůsobovat novým výzvám a příležitostem.


\section{Přidání bloků}
Rozšíření sbírky o více bloků a jejich verzí umožňuje uživatelům přizpůsobit si aplikace s větší flexibilitou a znovupoužitelností. Bloky mohou být navrženy tak, aby pokrývaly širokou škálu běžných i specifických případů, od navigačních prvků po složité sekce.

\section{Přidání komponent}
Dodatek nových komponent do knihovny by měl reflektovat aktuální trendy ve webovém designu, zatímco zároveň by měl pokrývat nové požadavky uživatelů. Například integrace komponent pro pokročilé uživatelské interakce jako jsou drag-and-drop prvky, pokročilé formulářové ovládací prvky a interaktivní grafy.

\section{Přidání ukázkových stránek/aplikací}
Vytvoření sad ukázkových stránek nebo aplikací využívajících komponent poskytuje uživatelům možnost rychlejšího vývoje. Tyto stránky mohou zahrnovat různé scénáře použití od jednoduchých landing stránek po složitější platformy.

\section{Přidání možností konfigurace}
Umožnění uživatelům konfigurovat komponenty pomocí vlastního konfiguračního souboru nabízí vyšší úroveň personalizace a kontrolu. Uživatelé již nyní mohou definovat globální nastavení, jako jsou motivy, typografie a barvy, které budou automaticky aplikovány na všechny komponenty. Nicméně někteří uživatelé mohou chtít konfigurovat např. umístění komponent a bloků a jejich prefix. Prozatím pro konzisteci napříč projekty není možné tyto nastavení měnit, avšak v budoucnu je možné přehodnotit tuto možnost.

\section{Figma Kit}
Momentálně je Figma Kit spíše statický, jeho aktualizace o responzivní design a interaktivní prvky umožní designérům a vývojářům lepší spolupráci a rychlejší iteraci designů. Tato rozšíření mohou zahrnovat varianty komponent pro různé obrazovky a zařízení, interaktivní stavy pro hover a kliknutí, a integrované prototypování funkcí. Poskytnutí těchto nástrojů v Figma kitu umožní týmům efektivně testovat a upravovat UI/UX designy před jejich implementací.

\section{Testy komponent}
Automatizované testování může být zajímavé rozšíření pro zajištění kvality a udržitelnosti knihovny komponent. Použití robustního testovacího frameworku, který umožňuje jak jednotkové testy, tak integrační a end-to-end testy, značně zvyšuje spolehlivost a snižuje pravděpodobnost chyb při budoucích aktualizacích komponent. Použití nástrojů jako @nuxt/test-utils nebo Cypress poskytuje komplexní pokrytí testů a integrovanou podporu pro simulaci uživatelských interakcí a asertace stavu UI. Navrhnout testovací scénáře, které mapují běžné a okrajové případy užití komponent, zvyšuje důvěru v kód a usnadňuje kontinuální integraci a nasazení. Na druhou stranu je otázka, jak tyto testy integrovat do koncových aplikací a zda je to vůbec možné.

\section{Vytváření stylů}
Vývoj \uv{Theme Creator} nástroje, který bude integrován přímo do dokumentace, umožní uživatelům v reálném čase vizualizovat a upravovat vzhled komponent. Tento nástroj by měl nabídnout intuitivní rozhraní pro výběr barev, fontů, zaoblení hran a dalších stylistických prvků. Takový nástroj značně usnadní proces designu a vývoje tím, že poskytne okamžitou zpětnou vazbu na změny provedené v tématu a umožní uživatelům exportovat výsledné styly pro použití ve svých projektech.
