%! Author = Vojta
%! Date = 21.1.2024

\chapter{Závěr}

Cílem této diplomové práce bylo analyzovat současný stav UI knihoven a na základě toho navrhnout a implementovat vlastní řešení. V průběhu práce se podařilo vytvořit komplexní ekosystém, který zahrnuje několik důležitých součástí.

Mezi hlavní části tohoto ekosystému patří:

\begin{description}
  \item[Dokumentace] Dokumentace je dostupná online na adrese \url{https://ui.vojtamoravec.cz}, což umožňuje snadnou dostupnost a aktualizaci informací pro uživatele knihovny.
  \item[Knihovna komponent ve Figmě] Na adrese \url{https://www.figma.com/community/file/1331024596435699926/v-moravec-ui} se nachází knihovna komponent, která je navržena tak, aby usnadnila designérům a vývojářům spolupráci a integraci návrhů do kódu.
  \item[CLI nástroj] Pomáha s instalací a správou komponent v rámci projektu.
\end{description}

V průběhu práce byly identifikovány a dokumentovány prvky, které ovlivňují design a implementaci komponent a samotné dokumentace. Byly popsány kroky vedoucí k navržení sbírky znovupoužitelných Nuxt komponent, které reflektují nejnovější trendy ve vývoji webových aplikací. Návrh a implementace zahrnovaly moderní frontend technologie a nástroje, které podporují celý životní cyklus vývoje od návrhu po nasazení a údržbu.

Testování provedené v rámci této práce odhalilo jak silné stránky, tak oblasti pro zlepšení kolekce komponent. Dokumentace byla obecně hodnocena kladně, ale bylo také jasné, že existuje prostor pro její zlepšení, zejména v ukázkách možností komponent. Jednoduchá rozšiřitelnost a flexibilita komponent byly hodnoceny jako důležité prvky kolekce, které umožňují snadnou integraci a rychlý vývoj webových stránek.

V závěru je třeba zdůraznit, že práce na této komponentové knihovně není ukončena. Plánují se další vylepšení a rozšíření, aby byla zajištěna jeho dlouhodobá udržitelnost a užitečnost pro vývojáře. Budoucí rozšíření zahrnují přidání nových bloků, komponent, ukázkových stránek a aplikací, možnosti konfigurace, Figma Kitu a testování komponent, což vše přispěje ke zlepšení kvality a uživatelského zážitku.

Tato diplomová práce tak poskytuje pevný základ pro další vývoj a rozšiřování komponentové knihovny, která bude sloužit jako užitečný nástroj pro vývojáře při vytváření moderních webových aplikací.
