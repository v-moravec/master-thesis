%! Author = Vojta
%! Date = 21.1.2024

\chapter{Technologie}
\label{chap:technologie}

\section{Nuxt}
Nuxt je moderní a výkonný open-source framework založený na Vue, který je navržen pro intuitivní a efektivní vývoj webových aplikací a stránek.
Tento framework se vyznačuje tím, že automaticky řeší mnoho opakujících se úkolů vývojářů, což jim umožňuje soustředit se na tvorbu samotné aplikace.
Nuxt využívá konvence a strukturované adresářové uspořádání pro automatizaci procesu vývoje a nabízí možnost vlastní konfigurace a přizpůsobení výchozích chování.

Nuxt má narozdíl od čistého Vue schopnosti server-side renderingu (SSR). Tato funkce zajišťuje rychlejší načítání stránek a lepší SEO optimalizaci,
protože celý obsah stránky je serverem vygenerován dříve, než začne běžet klientský JavaScript. Nuxt tento přístup kombinuje v takzvaném hybrid renderingu a má
tedy výhody tradičního server-side renderingu s interaktivitou a pokročilými uživatelskými rozhraními single-page aplikací (SPA). \cite{NuxtRenderingModes}

\section{Tailwind}
Tailwind je moderní a oblíbený CSS framework, který se zaměřuje na tzv. utility-first přístup. Tento přístup umožňuje vývojářům rychle a efektivně vytvářet
vlastní komponenty a designy s použitím nízkoúrovňových CSS tříd, které jsou přímo integrovány do HTML markupu.

Jednou z hlavních výhod Tailwind je možnost psát méně vlastního CSS. Vývojáři mohou využívat předdefinované třídy, jako jsou flexbox a padding utility, což znamená,
že většina stylů je opakovaně použitelná a zřídka je potřeba psát nové CSS. Tailwind také eliminuje potřebu vymýšlet názvy tříd, protože vývojáři vybírají třídy z
předdefinovaného designového systému. To znamená, že nemusí přemýšlet o "dokonalých" názvech tříd pro určité styly a komponenty nebo si pamatovat složité názvy. \cite{TailwindUtilityFirst}

Tailwind je známý svou flexibilitou a kontrolou nad tím, jak aplikace vypadá, což poskytuje větší prostor pro vytváření jedinečných webů. Na rozdíl od jiných CSS frameworků,
jako je Bootstrap nebo Materialize, Tailwind nenabízí plně stylované komponenty, jako jsou tlačítka nebo dropdown menu. Místo toho nabízí užitečnostní třídy,
které umožňují vytvořit vlastní opakovaně použitelné komponenty.

Přestože Tailwind nabízí mnoho výhod, může být jeho použití obtížné pro ty, kteří nejsou zkušení s CSS, a může způsobovat zmatek kvůli množství informací uložených v HTML souboru.
Navíc při instalaci Tailwindu jsou výchozí CSS styly odstraněny, takže je nutné je pro všechny elementy znovu vytvořit.

% \subsection{Body pro technologie}
% \begin{itemize}
%     \item Vue
%     \subitem Vue 3
%     \subitem script setup
%     \subitem Options vs Composition API
%     \item Nuxt
%     \item HeadlessUI
% \end{itemize}
