%! Author = Vojta
%! Date = 21.1.2024

\chapter{Analýza}

\section{Kvalitativní průzkum}
Vzhledem k tomu, že se jedná o projekt, který je určený hlavně pro využití autora, není kvalitatní průzkum příliš relevantní.
Místo toho se práce zaměřuje na porovnání existujících řešení, jejich výhod a nevýhod a názorů komunity.
Nicméně v jedné z následujících kapitol jsou výsledky z testování a výčet následných úprav.

\section{Existující řešení}

Existující řešení lze rozdělit do dvou skupin. První skupinou jsou knihovny, které nabízejí sadu komponent jako balíček, který
se do projektu přidá jako závislost. Poté se komponenty importují a používají v kódu. Jejich stylování lze upravit, ale většinou
pouze pomocí předpřipravených proměnných nebo limitovaným rozhraním. Úprava jejich funkcionality je většinou velmi omezená. Jakmile
chce uživatel upravit větší části, musí přistoupit k ohybání kódu, což často vede k neudržitelnému kódu.

Druhou skupinou jsou knihovny, které nejsou připraveny pro použití jako balíček. Jejich kód je dostupný pro zkopírování do projektu,
popř. mají dostupné CLI, které umožňuje komponenty přidat pomocí jednoduchého příkazu. Tato skupina míří na lepší znovupoužitelnost kódu
a zároveň na vyšší možnosti rozšíření a úprav.

\clearpage

\subsection{Knihovny s principem balíčkování}

\subsubsection{NuxtUI}
Jako zástupce knihoven na principu balíčkování jsem si vybral NuxtUI. Tuto knihovnu jsem využil na několika projektech, takže jsem si
dokázal udělat představu o jejích výhodách a nevýhodách. Líbilo se mi, že nabízí velké množství komponent, které jsou propracované a hezky
nastylované. Na druhou stranu jsem se často potýkal s tím, že jsem chtěl upravit funkčnost komponenty, ale nebylo to možné. Většinou jsem
musel přistoupit k vytvoření ohnutí ruzných částí. Tyto úpravy byly často velmi složité a neudržitelné.

\begin{itemize}
    \item \textbf{Výhody}
    \begin{itemize}
        \item Velké množství komponent
        \item Propracované styly
        \item Připravené pro použití
    \end{itemize}
    \item \textbf{Nevýhody}
    \begin{itemize}
        \item Omezené možnosti úprav
        \item Složité úpravy stylů

        \item Složité úpravy funkcionality
    \end{itemize}
\end{itemize}

\subsection{Knihovny s principem vlastnění kódu}
Knihovny, kde výsledný kód vlastníte většinou obsahují komponenty, které jsou jednodušší. Jejich výhodou je, že je lze snadno upravit, protože
kód není schovaný v balíčku. Nevýhodou je, že některé složitejší operace si uživatel musí vytvořit sám. Je tu ale jedna vyjímka, na kterou se chci
zaměřit, protože na podobném principu chci stavět i svojí kolekci.

% https://www.radix-ui.com/primitives/docs/overview/introduction
\subsubsection{Radix UI}
RadixUI představuje open-source knihovnu UI komponent, která je zaměřena na tvorbu kvalitních a přístupných designových systémů a webových aplikací.
Jde především o Radix Primitives, což jsou nízkoúrovňové UI komponenty s důrazem na přístupnost a možnost úprav vývojářů za cílem vytvoření vlastní knihovny.
Tyto komponenty lze využívat buď jako základní vrstvu designového systému nebo je postupně implementovat do stávajících projektů. \cite{RadixUIPrimitives}

\subsubsection{shadcn/ui}
Shadcn/ui je inovativní open-source knihovna UI komponent, která je navržena tak, aby vylepšila webový vývoj, zejména pro projekty využívající React.
Hlavní předností této knihovny je její lehkost, díky čemuž se snadno integruje do projektů bez nutnosti zatěžujících závislostí. Knihovna staví
zejména na komponentách Radix UI, které kladou důrat na přístupnost, což zajišťuje, že komponenty jsou inkluzivní a použitelné pro všechny uživatele. \cite{ShadcnUI}

Dalším významným aspektem knihovny Shadcn/ui je integrace s frameworkem Tailwind CSS, který poskytuje efektivní a přívětivé prostředí pro vývojáře,
preferující tento CSS framework orientovaný na utilitu. Výhody Tailwind CSS jsou popsány v kapitole věnované technologiím. Tato integrace umožňuje snadnou úpravu
a rozšíření stylů. Shadcn/ui vyniká svou snadnou použitelností a detailní kontrolou nad komponentami. Vývojáři mohou přímo přistupovat ke zdrojovému kódu
jednotlivých komponent, což umožňuje jejich efektivní úpravy pro specifické případy užití a požadavky aplikace.

Ruční instalace nebo kopírování každé komponenty může být pro některé vývojáře zdlouhavé, zejména pro ty, kteří jsou zvyklí importovat komponenty z balíčků.
Ačkoli přímý přístup ke kódu komponent prospívá modularitě a rozšiřitelnosti, může vést k rozšíření objemu kódu.

% Headless UI

\begin{itemize}
    \item \textbf{Výhody}
    \begin{itemize}
        \item Jednoduché úpravy
        \item Kód vlastní uživatel
        \item Aktualizace knihovny neovlivní kód
    \end{itemize}
    \item \textbf{Nevýhody}
    \begin{itemize}
        \item Omezená funkcionalita
        \item Větší objem spravovaného kódu
    \end{itemize}
\end{itemize}


% \subsection{Knihovny}

% \section{Body pro analýzu}

% \subsection{Existující řešení}

% \begin{itemize}
%     \item Knihovny
%     \begin{itemize}
%         \item NuxtUI
%     \end{itemize}
%     \item Princip kopírování kódu
%     \begin{itemize}
%         \item Radix UI
%         \item shadcn/ui
%         \item TailwindUI
%         \item HeadlessUI
%         \item Adobe Spectrum (React Aria)
%     \end{itemize}
%     \item CLI
%     \begin{itemize}
%         \item Create T3 App
%         \item VS Code Extension
%     \end{itemize}
    
%     \item dspace
    
%     \item Kvalitativní průzkum
% \end{itemize}
